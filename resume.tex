\documentclass[letterpaper,11pt]{article}

\usepackage{latexsym}
\usepackage[empty]{fullpage}
\usepackage{titlesec}
\usepackage{marvosym}
\usepackage[usenames,dvipsnames]{color}
\usepackage{verbatim}
\usepackage{enumitem}
\usepackage[hidelinks]{hyperref}
\usepackage{fancyhdr}
\usepackage{tabularx}
\usepackage[english, russian]{babel}
\input{glyphtounicode}

\pagestyle{fancy}
\fancyhf{}
\fancyfoot{}
\renewcommand{\headrulewidth}{0pt}
\renewcommand{\footrulewidth}{0pt}

\addtolength{\oddsidemargin}{-0.5in}
\addtolength{\evensidemargin}{-0.5in}
\addtolength{\textwidth}{1in}
\addtolength{\topmargin}{-.5in}
\addtolength{\textheight}{1.0in}

\urlstyle{same}

\raggedbottom
\raggedright
\setlength{\tabcolsep}{0in}

\titleformat{\section}{
  \vspace{-4pt}\scshape\raggedright\large
}{}{0em}{}[\color{black}\titlerule \vspace{-5pt}]

\pdfgentounicode=1

\newcommand{\resumeItem}[1]{
  \item\small{
    {#1 \vspace{-2pt}}
  }
}

\newcommand{\resumeSubheading}[4]{
  \vspace{-2pt}\item
    \begin{tabular*}{0.97\textwidth}[t]{l@{\extracolsep{\fill}}r}
      \textbf{#1} & #2 \\
      \textit{\small#3} & \textit{\small #4} \\
    \end{tabular*}\vspace{-7pt}
}

\newcommand{\resumeSubSubheading}[2]{
    \item
    \begin{tabular*}{0.97\textwidth}{l@{\extracolsep{\fill}}r}
      \textit{\small#1} & \textit{\small #2} \\
    \end{tabular*}\vspace{-7pt}
}

\newcommand{\resumeProjectHeading}[2]{
    \item
    \begin{tabular*}{0.97\textwidth}{l@{\extracolsep{\fill}}r}
      \small#1 & #2 \\
    \end{tabular*}\vspace{-7pt}
}

\newcommand{\resumeSubItem}[1]{\resumeItem{#1}\vspace{-4pt}}

\renewcommand\labelitemii{$\vcenter{\hbox{\tiny$\bullet$}}$}

\newcommand{\resumeSubHeadingListStart}{\begin{itemize}[leftmargin=0.15in, label={}]}
\newcommand{\resumeSubHeadingListEnd}{\end{itemize}}
\newcommand{\resumeItemListStart}{\begin{itemize}}
\newcommand{\resumeItemListEnd}{\end{itemize}\vspace{-5pt}}

\begin{document}

\begin{center}
    \textbf{\Huge \scshape Максим Минеев} \\ \vspace{1pt}
    \small +7-901-413-4083 $|$ \href{mailto:maksim.mineeff@gmail.com}{\underline{maksim.mineeff@gmail.com}} $|$ 
    \href{https://github.com/mineevmaxim}{\underline{github.com/mineevmaxim}}
\end{center}


%-----------EDUCATION-----------
\section{Образование}
  \resumeSubHeadingListStart
    \resumeSubheading
      {Университет УрФУ}{Екатеринбург}
      {Программная инженерия}{Сент. 2022 -- Июнь 2026}
  \resumeSubHeadingListEnd


%-----------EXPERIENCE-----------
\section{Опыт}
  \resumeSubHeadingListStart

      \resumeSubheading
      {C\# Разработчик}{Янв. 2024 -- Июль. 2024}
      {66bit}{Удаленно}
      \resumeItemListStart
        \resumeItem{Поддерживал и улучшал кодовую базу существующих микросервисов на С\#}
        \resumeItem{Внедрил CI/CD для удобства разработки и деплоя}
        \resumeItem{Обеспечивал чистую архитектуру и высокую надежность системы}
        \resumeItem{Создавал микросервисы для работы с пользовательскими данными}
        \resumeItem{Реализовал интеграцию с внешними API для расширения функционала платформы}
      \resumeItemListEnd
      
  \resumeSubHeadingListEnd


%-----------PROJECTS-----------
\section{Проекты}
    \resumeSubHeadingListStart

    \resumeProjectHeading
      {\href{https://github.com/mineevmaxim/clean-cone}{\textbf{Markdown processor}} $|$ \emph{C\#, NUnit, FluentAssertions, TDD}}{Нояб. 2024 -- Дек. 2024}
      \resumeItemListStart
        \resumeItem{Использовал стиль Test Driven Desing для удобства разработки и тестирования}
        \resumeItem{Разработал лексер для преобразования Markdown в токены}
        \resumeItem{Реализовал парсер токенов в AST-дерево  Markdown}
        \resumeItem{Разработал транслятор MD AST-дерева в HTML}
        \resumeItem{Полностью покрыл код тестами с помощью NUnit, FluentAssertions}
      \resumeItemListEnd

    \resumeProjectHeading
        {\href{https://github.com/mineevmaxim/tdd}{\textbf{Tags cloud}} $|$ \emph{C\#, NUnit, FluentAssertions, TDD, DDD, SixLabors.ImageSharp, DI, NHunspell}}{Окт. 2024}
      \resumeItemListStart
        \resumeItem{Разработал консольный интерфейс для генерации облака тегов}
        \resumeItem{Реализовал архитектуру Domain Driven Design для удобства расширения и тестирования}
        \resumeItem{Внедрил Dependency Injection для гибкого управления зависимостями}
        \resumeItem{Использовал NHunspell для предобработки слов}
        \resumeItem{Разработал алгоритм раскладки слов по спирали}
        \resumeItem{Воспользовался библиотеками SixLabors для генерации итогового изображения облака тегов}
      \resumeItemListEnd

    \resumeSubHeadingListEnd

%-----------COURSES-----------
\section{Курсы}
    \resumeSubHeadingListStart

    \resumeProjectHeading
        {\href{https://kontur.ru/education/programs/courses_backend/2024}{\textbf{Школа промышленной разработки, СКБ Контур}} $|$ \emph{Rest API, multithreading, безопасность, оптимизация}}

    \resumeProjectHeading
        {\href{https://ulearn.me/Course/cs2}{\textbf{Проектирование на C\#}} $|$ \emph{DDD, TDD, чистый код, рефлексия, FluentAPI, DI, исключения}}

    \resumeProjectHeading
        {\href{https://ulearn.me/Course/basicprogramming}{\textbf{Основы программирования 1, 2}} $|$ \emph{C\#, память, ООП, тестирование}}

    \resumeProjectHeading
        {\href{https://ulbitv.ru/frontend}{\textbf{Продвинутый frontend}} $|$ \emph{React, RTKQuery, Storybook, Jest, i18n, CI/CD, Cypress, pre-commit hooks}}

    \resumeSubHeadingListEnd

%-----------PROGRAMMING SKILLS-----------
\section{Навыки}
 \begin{itemize}[leftmargin=0.15in, label={}]
    \small{\item{
     \textbf{Языки программирования}{ C\#, TypeScript, Python, PHP, SQL, PostgreSQL } \\
     \textbf{Инструменты}{: Git, Docker } \\
     \textbf{Библиотеки}{: ASP.NET Core, NUnit, FluentAssertions,  React, Angular }
    }}
 \end{itemize}

\end{document}
